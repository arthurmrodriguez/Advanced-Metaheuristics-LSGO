\chapter{Introducción}

Desde sus inicios, la computación evolutiva ha sido capaz proponer soluciones efectivas a una gran cantidad de problemas de optimización a través de algoritmos bio-inspirados, aquellos que se basan en comportamientos o procesos puramente naturales. Algoritmos Evolutivos (EA, Evolutionary Algorithms), de Nubes de Partículas (PSO, Particle Swarm Optimization), de Evolución Diferencial (DE, Differential Evolution) de Colonias de Hormigas (ACO, Ant Colony Optimization), responden de manera satisfactoria al ser aplicados a problemas de escalas pequeñas-medianas.

Sin embargo, cuando el tamaño del problema se situa en el rango de los cientos o miles de variables, el espacio de soluciones aumenta de manera exponencial conforme crece el número de éstas, lo que repercute directamente en un aumento significativo de la complejidad del problema. Estas condiciones hacen que para una técnica, anteriormente considerada como efectiva, sea mucho más difícil el encontrar una solución óptima para el problema en cuestión.

Con miras a solventar estas dificultades, surge una nueva vertiente de propuestas dedicadas a la resolución de problemas de optimización de escalas igual o superiores a las mil variables, lo que se conoce como\textit{ Evolutionary Large Scale Global Optimization} o \textbf{ELSGO}\cite{ELSGOI}. Estas nuevas técnicas son, principalmente, producto de los más importantes congresos en computación evolutiva del mundo, como el Congress of Evolutionary Computation (IEEE CEC) o el World Congress on Computational Intelligence (WCCI).

\section{Motivación}

Las propuestas disponibles actualmente muestran resultados satisfactorios al ser evaluadas en función de benchmarks preestablecidos y bien definidos. No obstante, su aplicación real normalmente no trasciende más allá de la propia competición para la que fueron diseñados. Este hecho supone la principal motivación de este trabajo: el estudio de la efectividad y eficiencia de estos algoritmos cuando son aplicados a un problema médico real como lo es, en este caso, \textbf{la optimización de los datos de un electroencefalograma}, EEG\cite{EEG} por sus siglas en inglés.

Como se ha mencionado anteriormente, la optimización de un electroencefalograma es un problema que no guarda relación con ninguno de los benchmarks conocidos hasta la fecha, al ser un problema cuya \textbf{complejidad} es \textbf{superior tanto a nivel conceptual como en tiempo y en espacio}. Propuestas que conlleven la resolución de este problema de manera satisfactoria podrían reportar importantes avances en cuanto al diseño e implementación de \textbf{interfaces cerebro-ordenador} (BCI, \textit{brain-computer interfaces}) \cite{BCI} más potentes, fiables y eficientes.

Este tipo de dispositivos podría ser de ayuda en actividades y tareas del día a día donde la decodificación de un EEG cualitativo es crucial a la hora de reconocer estados cognitivos de orden superior, tales como emociones, memoria o planificación, aspectos que influyen directamente en la toma de decisiones críticas\cite{EvolutionaryBigOpt} en distintos entornos donde es necesaria una interacción y respuesta en tiempo real para garantizar el correcto desempeño de la actividad.

\section{Objetivos}

El objetivo principal de este Trabajo de Fin de Grado consiste en realizar un estudio comparativo de los algoritmos dedicados a resolver problemas de optimización con miles de variables, problemas del tipo \textbf{LSGO}, de forma que se pueda plantear un posible curso de acción o propuesta que sea de utilidad para resolver el problema médico real de la optimización de un electroencefalograma (EEG).

Los siguientes capítulos contendrán el estudio completo en cuestión: desde la definición y representación del problema hasta la selección de técnicas y procedimientos candidatas al estudio, así como del proceso de experimentación, la interpretación y valoración de los resultados obtenidos. Los objetivos principales se encuentran recogidos en los siguientes enunciados:

\begin{itemize}
	\item \textbf{Realizar la descripción y representación del problema del EEG en su totalidad.}
	\item \textbf{Analizar en profundidad los algoritmos y técnicas más prometedoras para la resolución del problema.}
	\item \textbf{Diseñar e implementar un proceso de experimentación riguroso y completo para cumplir los requisitos del estudio.}
	\item \textbf{Evaluar los resultados obtenidos y enunciar la propuesta de solución más adecuada en función de éstos.}
\end{itemize}

Una vez enunciados los objetivos, obtener una perspectiva clara y concisa del problema del EEG es crucial para comprender la magnitud del estudio y todo lo que conlleva, dado que compone la base sobre la que se sustenta éste y es la que marca el curso de acción a tomar en función de los requisitos y necesidades que plantee su resolución. 

\section{Descripción y representación del problema: Optimización de un EEG}








