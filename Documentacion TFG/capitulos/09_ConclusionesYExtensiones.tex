\chapter{Conclusiones y posibles extensiones}\label{cap:Conclusiones}

En este trabajo se ha estudiado cómo es el rendimiento y la escalabildad de las propuestas más novedosas y potentes del panorama de optimización en alta dimensionalidad. Para lograr este objetivo, se ha utilizado un benchmark con una formulación de un problema totalmente distinto a los que se proponen en los benchmarks más utilizados actualmente: la optimización de los datos de un electroencefalograma en un ámbito médico real.

Se han estudiado el diseño de estas propuestas y se han adaptado las mismas para ser capaces de procesar la función objetivo del problema propuesto, lo que derivaría en la posterior obtención de los resultados, a través de un proceso de experimentación exhaustivo, que servirían de base para sustentar la posible utilidad de estas técnicas de cara a una implantación en un entorno real.

Los resultados han sido analizados y se concluye que las técnicas elegidas del ámbito \textit{large scale global optimization} son muy robustas, escalables y potentes. Sin embargo, aún queda un largo camino por recorrer en materia LSGO, donde el principal inconveniente sigue siendo los excesivos tiempos de ejecución de estos algoritmos, aunque como se ha visto, distintas implementaciones utilizando distintas tecnologías podrían ser capaces de resolver estas deficiencias.

A criterio del alumno, no se conoce en la literatura actual del panorama de optimización en alta dimensionalidad un trabajo como el que se presenta en estas páginas. Es este precepto el que motivó la realización del mismo, donde se propone una hoja de ruta a seguir en cuanto a la optimización del problema del electroencefalograma se refiere, lo que puede derivar, ya sea a corto-medio o largo plazo, en una mejora sustancial de las técnicas empleadas actualmente para su optimización, lo que se traduciría en un uso mucho más extensivo y eficiente en determinadas áreas de conocimiento y para determinadas aplicaciones que así lo requieran.

Como posibles extensiones de este trabajo se propone un ajuste exhaustivo y extensivo de los parámetros de cada algoritmo siguiendo el enfoque del benchmark estudiado en este trabajo, con vistas a conseguir una posible mejora de las soluciones obtenidas. De forma análoga, la inclusión de más benchmarks formulados a partir de problemas reales aumentaría la diversidad de los estudios, lo que podría llevar a mejorar el rendimiento de las técnicas actuales.

Ampliar el presente estudio con más técnicas también puede formar parte de una extensión futura, donde se añadan implementaciones en lenguajes más eficientes de cara a preparar las propuestas a un entorno real donde las restricciones temporales juegan un papel clave. Realizar una mejora del propio benchmark del EEG con el fin de que sea más representativo de un problema médico real, añadiendo más variables y mayor ruido, para crear mayor incertidumbre y forzar a que las técnicas que busquen su resolución sean cada vez más sofisticadas, se considera también una posible extensión de este trabajo.

Concluye así por tanto este estudio, donde el alumno agradece toda la ayuda proporcionada por la institución en cuanto a los recursos para llevar a cabo este trabajo, así como al tutor del mismo por el apoyo condicional y por guiar el buen desarrollo de este Trabajo Fin de Grado.