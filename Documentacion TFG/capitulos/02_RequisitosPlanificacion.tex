\chapter{Tareas y Planificación del trabajo}\label{cap:Requisitos}

En este capítulo se propone el curso de acción para el desarrollo del trabajo. Este incluye las principales \textbf{tareas a desarrollar} por el alumno en cuanto a los \textbf{objetivos} y la \textbf{propuesta}, así como la \textbf{planificación} de las mismas, entendida como la previsión en relación a recursos necesarios, sus costes derivados, optimización del tiempo de proceso, necesidades de potencia y espacio a nivel computacional, la estructuración de la carga de trabajo y por último la \textbf{metodología} idónea para elaborar este análisis

\section{Tareas del proceso de investigación}\label{section:Tareas}

La naturaleza investigativa de este trabajo implica el cumplimiento de una serie de tareas de cara a establecer la base sobre la que se sustenta el mismo; éstas se encuentran representadas en los siguientes enunciados y \textbf{componen las tareas más fundamentales }a llevar a cabo para completar satisfactoriamente el estudio. Como norma, se definirá una jornada de trabajo de entre 6 y 8 horas diarias para definir el tiempo total necesario para completar cada tarea.

\begin{enumerate}
	\item \label{tarea1} \textbf{Estudiar de forma exhaustiva el problema del EEG}: \textbf{comprender} la naturaleza más primitiva de la proposición en todos los niveles, estudiar su \textbf{utilidad a escala real} y los \textbf{beneficios} que trae consigo su resolución para la sociedad. Esta tarea se satisface a lo largo del capítulo \ref{cap:QEEG} y se estima alrededor de 2 semanas en completar esta fase de forma adecuada, allanando el camino para posteriores tareas, por lo que la estimación de tiempo total no supera las 120 horas. 
	
	\item \label{tarea2} \textbf{Revisar la literatura actual}: una vez conocido el problema en profundidad, el siguiente paso consiste en hacer una \textbf{revisión de la literatura del panorama LSGO}. Conocer en mayor medida las propuestas anteriormente publicadas en cuanto a la resolución del problema descrito, con el objetivo de identificar aquellas técnicas más prometedoras de cara a obtener resultados satisfactorios aplicando el benchmark EEG. La revisión, motivada por sugerencias del tutor y su experiencia en este campo, \textbf{no será mayor de 60h} y se desarrolla en el capítulo \ref{cap: EstadoArte}
	
	\item \label{tarea3} \textbf{Seleccionar los principales algoritmos y técnicas a incluir en el trabajo}: teniendo en cuenta la revisión efectuada en la tarea anterior, junto con recomendaciones del tutor, se \textbf{eligen las tecnologías más prometedoras} de la literatura que formarán parte del estudio que se desarrolla a lo largo de este trabajo. El capítulo \ref{cap:Analisis} recoge esta tarea junto con la siguiente, donde la actual tarea no ocupará mas de 1 o 2 días de trabajo, menos de 16h.
	
	\item  \label{tarea4}\textbf{Analizar las propuestas elegidas con el mayor nivel de detalle posible}: identificar los \textbf{puntos fuertes de cada implementación} y de las técnicas empleadas, con el objetivo de construir \textbf{juicios preliminares} en relación a la satisfacción de la propuesta, que serán finalmente validados en el proceso de experimentación. Se recoge este análisis en el capítulo \ref{cap:Analisis}, donde se espera una gran carga de trabajo que, al alza, ocuparía entre un mes y medio a dos, cerca de 480h.
	
	\item \label{tarea4-1} \textbf{Obtener las implementaciones de las técnicas seleccionadas}: debido a las restricciones de eficacia y eficiencia que impone el problema, se requieren implementaciones sin fisuras y lo más testadas y completas posibles, por lo que se optará por \textbf{obtener una copia de las implementaciones} de los algoritmos a través de sus autores y/o sus respectivos repositorios, así como a través del tutor de este trabajo. El proceso de obtención de estas técnicas está implícito en el capítulo \ref{cap:Implementacion}, donde se responderá de forma conjunta a ésta y la siguiente tarea. La estimación es que no ocupe más de 15h en total, aunque está claramente supeditado a la respuesta de los autores o la disponibilidad de las implementaciones en los repositorios. 
		
	\item \label{tarea5} \textbf{Implementar la adaptación de las técnicas al problema del EEG}: obtenidas las implementaciones particulares de cada técnica y teniendo en cuenta las herramientas utilizadas por los autores, es necesaria una \textbf{adaptación de la función objetivo} para que pueda ser procesada por los algoritmos de forma correcta. Una \textbf{configuración personalizada} será necesaria para acoplar esta función con las exigencias a nivel de código fuente de cada técnica. Se documenta esta tarea en el capítulo \ref{cap:Implementacion}, donde se estima cerca de una semana (60h) para cada técnica elegida y la validación de resultados preliminares, previo al proceso experimental completo. Total: 240-300h.
	
	\item  \label{tarea6}\textbf{Diseñar un proceso de experimentación completo}: para dotar de validez a las propuestas y obtener resultados representativos, se hace indispensable \textbf{planificar una fase de experimentación} robusta y \textbf{acotar la salida} que se quiere obtener para posteriormente realizar un análisis sólido y objetivo de los resultados. Se desarrolla a lo largo del capítulo \ref{cap:DisenioExp} y, tomando como referencia la literatura de los benchmarks y como se conducen normalmente los experimentos, 50h deberían ser suficientes para completar esta tarea.
	
	\item \label{tarea7} \textbf{Realizar los experimentos propuestos}: llevar a cabo los experimentos es crucial para obtener los resultados que motivarán las conclusiones expresadas posteriormente. Dado que depende directamente del tiempo de respuesta de cada algoritmo, la experimentación efectiva (hasta obtener los resultados) no debería ocupar más de una semana, cerca de 60h. Esta tarea no requiere documentación expresa, dado que es en la siguiente tarea donde se mostrarán los resultados obtenidos.
	
	\item  \label{tarea8} \textbf{Crear las tablas y gráficas de los resultados obtenidos}: un proceso  indispensable que permitirá tener los \textbf{datos ordenados} y que sea \textbf{fácil identificar} aquellas técnicas con mejores rendimientos. Se estima poco mas de 30h para tener completada esta fase, siendo el preámbulo de la posterior tarea que, aunque se desarrollen ambas en el capítulo \ref{cap:PruebasResultados}, se añade como una tarea por separado
	
	\item \label{tarea9} \textbf{Analizar los resultados obtenidos de forma objetiva}: de acuerdo con las soluciones alcanzadas, utilizar todas las herramientas de representación de datos que sean necesarias para \textbf{decretar un veredicto con el mayor grado de precisión posible}, así como valerse de la información resumida de la tarea anterior. En 24h se podría tener un veredicto preliminar que precisará de consultas con el tutor. El capítulo \ref{cap:PruebasResultados} recoge la realización de esta tarea.
	
	\item  \label{tarea10} \textbf{Destacar las conclusiones y extensiones futuras}: enunciar las conclusiones de este trabajo, en forma de resumen de todo el proceso realizado, así como contemplar posibles extensiones en materia de investigación dentro de la actual disciplina. El capítulo \ref{cap:Conclusiones} recoge estos enunciados finales y no deberían superarse las 24h de trabajo.
	
	\item  \label{tarea11} \textbf{Documentar el trabajo realizado}: esta es la única tarea que será realizada, de forma incremental, a lo largo de todo el desarrollo de este trabajo, por lo que se estima un tiempo de finalización total similar al que ocupa la suma del tiempo de cada tarea, dado que durante el desarrollo de cada una se documentará el proceso de forma paralela.
\end{enumerate}

Una vez especificadas las tareas a desarrollar a lo largo del trabajo, el siguiente paso consiste en planificar el proceso de estudio de forma que se contemplen, aunque sea de forma preliminar, los recursos y necesidades indispensables a las que habrá que hacer frente a lo largo del desarrollo, para tener una primera perspectiva de lo que implica llevar a cabo la investigación actual.

\section{Planificación del trabajo} 

Los aspectos principales a considerar en este apartado están relacionados con los \textbf{recursos} disponibles y aquellos que son necesarios, así como la \textbf{distribución de la carga de trabajo} en términos, principalmente, de tiempo y potencia de cómputo. Considerar ambos factores implica realizar una \textbf{estimación de tiempos, costes y requerimientos} para la realización del trabajo.

\subsection{Recursos disponibles y necesarios}

El alumno hará uso de su propio ordenador portátil MacBook Pro del año 2012 con procesador Intel Core i7 a 2,9 GHz, con 16 GB de memoria RAM DDR3 y una tarjeta gráfica Intel HD Graphics de 1,5 GB. Este equipo será más que suficiente para afrontar las tareas derivadas de la \textbf{investigación y documentación}, para las cuales también será necesario tener acceso a una \textbf{red de internet} de banda ancha para consultar toda la bibliografía y documentación requerida, red de la que el alumno dispone en su domicilio particular.

Acceder a \textbf{documentación, artículos científicos y libros}, ya sea a nivel de Internet o de bibliotecas virtuales, será posible gracias a los convenios que tiene la Universidad de Granada con bases de datos tales como Scopus\cite{SCOPUS} o IEEE, que proporcionan bibliografía fiable y contrastada, imprescindible en este trabajo, y se accederá a ellas a través de una conexión VPN a la red de la Universidad. El alumno se valdrá también de una libreta donde documentará de forma escrita aquellas fases o hitos que considere necesarios.

Para la \textbf{fase de implementación}, donde se validará la adaptación de los algoritmos, bastará con el ordenador portátil disponible y las principales herramientas software como distintas IDEs, procesadores de texto y demás tecnologías, así como los recursos energéticos que proporciona el lugar donde reside el alumno, medios que serán igualmente suficientes para los \textbf{procesos finales de evaluación y conclusión}. 

Sin embargo, dada la \textbf{dimensionalidad del problema y la complejidad de los experimentos}, será necesaria una mayor potencia de cómputo para llevar a cabo las pruebas pertinentes. Para ello, el alumno se servirá del \textbf{Clúster Hercules del CITIC de la UGR} \cite{HERCULES}, que \textit{``posee 46 nodos, cada uno de ellos equipado con un procesador Intel Core i7 930 a 2.8 GHz, 24 GB de RAM y HDD SATA2 de 1TB"}. Funciona a través de un sistema de colas SLURM\cite{SLURM} que facilita la creación y ejecución de trabajos en el clúster. El acceso a este clúster se realizará a través de un usuario y contraseña proporcionados por el equipo de SysOp de la UGR.

El alumno por tanto asumirá únicamente los costes derivados de la investigación cuando utilice los \textbf{recursos propios} y las herramientas de las que dispone, por lo que no se tienen en consideración aquellos que asume la propia Universidad, como cuando se utiliza el Cluster Hércules. 
\subsection{Estimaciones y presupuesto}

Teniendo en cuenta el tiempo estimado para cada tarea que se detalla en la sección \ref{section:Tareas} se estima un total de \textbf{6 meses} en llevar a cabo el trabajo, donde serán las tareas de documentación las que en un principio requerirán mayor tiempo, sobre todo las relacionadas con las \textbf{tareas \ref{tarea1}, \ref{tarea2}, \ref{tarea3} y\ref{tarea4}}, dado que son las que sustentan toda la base del estudio y requieren ser precisadas de forma exhaustiva para descartar la existencia de cualquier tipo de incongruencia o error.

El proceso de \textbf{adaptación de las técnicas} al problema del EGG (tarea \ref{tarea5}) está subordinado a las particularidades de cada implementación y puede complicarse por situaciones que actualmente escapan al conocimiento del alumno, pero es evidente que resolver las tareas \ref{tarea3} y \ref{tarea4} despeja el camino a seguir y una vez realizada la adaptación sobre uno de los candidatos, se asume que el proceso será similar para los algoritmos restantes. 

El \textbf{diseño e implementación experimental} (tareas \ref{tarea6} y \ref{tarea7}) quizá sean la tareas con mayor incertidumbre en cuanto a estimaciones, dado que está ligada de forma directa a los \textbf{procesos de validación previos} donde se evaluarán principalmente los tiempos de ejecución de los algoritmos. Cuando finalmente sean sometidos a test, estos tiempos supondrán la gran parte del tiempo total que se empleará en esta fase. Al contar con recursos como el Clúster Hércules, los tiempos de respuesta de los algoritmos frente a los experimentos no deberían suponer un gran problema, independientemente de lo que se tarde en ellos, por lo que las estimaciones anteriormente mencionadas deben ser suficientes. El resto de tareas se desarrollarán de acuerdo con las estimaciones temporales mencionadas.

En resumen, la tabla siguiente muestra como se distribuye el presupuesto en relación a las tareas mencionadas en la sección anterior, tomando un precio por hora de trabajo de \textbf{ocho euros}; en base al número de horas y el precio por hora, se puede estimar un presupuesto aproximado de lo que conlleva realizar este trabajo. Para las tareas que ocupen tiempos más variables, se tomará el mayor número de horas para evitar subestimaciones.

\begin{table}[H]
	\centering
	\resizebox{\textwidth}{!}{
		$\begin{tabular}{ *{4}{c}}
		\toprule
		\textbf{Tareas} & \textbf{Tiempo (h.)}  & \textbf{Presupuesto (euros)}\\
		\midrule 
		Estudio problema & 120 & 960\\ 
		Rev. Literatura & 60 & 480 \\ 
		Selección técnicas & 16 & 128 \\ 
		Análisis propuestas & 480 & 3840 \\ 
		Obtener impl. & 15 & 120 \\ 
		Adaptar técnicas & 300 & 2400 \\
		Diseño experimental & 50 & 400 \\
		Realizar exps. & 60 & 480\\
		Tablas y gráficas & 30 & 240 \\
		Analisis resultados & 24 & 192 \\
		Conclusiones & 24 & 192 \\
		\midrule 
		\midrule
		\textbf{Total} & 1179 &\textbf{9432}\\
		\bottomrule
		\end{tabular}$
	}
	\caption{Presupuesto estimado del trabajo}
	\label{tabla:Presupuesto}
\end{table}

\subsection{Metodología}

La metodología a seguir para llevar a cabo este trabajo será la de un desarrollo iterativo e incremental donde cada ciclo estará formado por una fase similar: la especificación de requisitos se realizará \textbf{a nivel de cada una de las tareas} que se esten desarrollando, el \textbf{diseño} se limitará a la \textbf{fase experimental}, la \textbf{implementación} casa con la adaptación de los algoritmos, la puesta en marcha de los experimentos y con el proceso de creación, modificación y adaptación de \textbf{código fuente}. La \textbf{documentación estará íntimamente ligada a cada uno de los procesos}, por lo que también se desarrolla de forma iterativa e incremental.

A lo largo de todo el proceso de desarrollo de la investigación, la \textbf{comunicación directa con el tutor} del trabajo será vital si se quiere consultar y/o validar cada una de las iteraciones realizadas y llevar un seguimiento del desarrollo del trabajo. Dada las situaciones particulares tanto del alumno como del tutor, se primará el uso del \textbf{correo electrónico} para el seguimiento del trabajo, siendo indispensables al menos \textbf{dos o tres reuniones más}, sin contar la primera, de cara a la culminación del proyecto.

Sin más aspectos que añadir en esta fase, en el capítulo siguiente se procederá al estudio en profundidad del problema del EEG, tarea que junto con la revisión de la literatura, supone el punto de partida del resto del trabajo, por lo que se precisa conocer con la mayor exactitud posible todos los componentes de la actual formulación.



