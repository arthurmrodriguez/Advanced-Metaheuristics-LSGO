\chapter*{}
%\thispagestyle{empty}
%\cleardoublepage

%\thispagestyle{empty}

\begin{titlepage}
 
 
\setlength{\centeroffset}{-0.5\oddsidemargin}
\addtolength{\centeroffset}{0.5\evensidemargin}
\thispagestyle{empty}

\noindent\hspace*{\centeroffset}\begin{minipage}{\textwidth}

\centering
\includegraphics[width=0.9\textwidth]{imagenes/logo_ugr.jpg}

%\textsc{ \Large PROYECTO FIN DE CARRERA\\[0.2cm]}
%\textsc{ INGENIERÍA EN INFORMÁTICA}\\[1cm]
% Upper part of the page
% 

 \vspace{2.5cm}


% Title

{\Huge\bfseries \myTitle\\
}
\noindent\rule[-1ex]{\textwidth}{3pt}\\[3.5ex]
\end{minipage}

\vspace{1.5cm}
\noindent\hspace*{\centeroffset}\begin{minipage}{\textwidth}
\centering

\textbf{Autor}\\ {\myName}\\[2.5ex]
\textbf{Director}\\
{\myProf}\\[1.5cm]
\includegraphics[width=0.4\textwidth]{imagenes/decsai}\\[0.1cm]
\textsc{\myDepartment}\\
%\textsc{---}\\
%Granada, mes de 201
\end{minipage}
%\addtolength{\textwidth}{\centeroffset}
\vspace{\stretch{2}}

 
\end{titlepage}






\cleardoublepage
\thispagestyle{empty}

\begin{center}
{\large\bfseries \myTitle}\\
\end{center}
\begin{center}
\myName\\
\end{center}

%\vspace{0.7cm}
\noindent{\textbf{Palabras clave}:  big optimization, electroencefalograma, metaheurísticas, algoritmos, big data, benchmark, LSGO, MAGA, MOS, SHADEILS, MLSHADE-SPA, DG2 }\\

\vspace{0.7cm}
\noindent{\textbf{Resumen}}\\

Desde sus albores, la computación evolutiva ha contribuido significativamente a la resolución de problemas del mundo real representados a través de funciones que requieren optimización. Sin embargo, el rendimiento de estas técnicas disminuye considerablemente cuando se enfrentan a un problema con miles de variables. En este proyecto se estudia la efectividad de las técnicas metaheurísticas actuales más avanzadas en el área de optimización de alta dimensionalidad frente a la formulación teórica de un problema médico real, como es la optimización de los datos de un electroencefalograma. Se desarrolla un estudio experimental exhaustivo que conforma la base del objetivo final, que consiste en evaluar la utilidad real de las propuestas más modernas del panorama actual en áreas de interés como la medicina, donde la decodificación de un electroencefalograma de forma eficaz y eficiente permitiría mejorar las interfaces cerebro-ordenador actuales e incluso potenciar su uso en entornos del mundo real donde se requiera la toma de decisiones críticas y respuestas en tiempo real.

\cleardoublepage


\thispagestyle{empty}


\begin{center}
{\large\bfseries \myTitleEng}\\
\end{center}
\begin{center}
\myName\\
\end{center}

%\vspace{0.7cm}
\noindent{\textbf{Keywords}: big optimization, electroencephalography, metaheuristics, algorithms, big data, benchmark, LSGO, MAGA, MOS, SHADEILS, MLSHADE-SPA, DG2}\\

\vspace{0.7cm}
\noindent{\textbf{Abstract}}\\

Since early days, evolutionary computation has undoubtedly contributed to solve several real-world problems represented as optimization functions. However, the performance of these techniques decreases significantly when applied over problems containing thousands of variables. In this proposal, the capabilities of the most advanced metaheuristic techniques of large scale global optimization are studied, facing the theoretical formulation of a medical real-world problem such as the optimization of electroencephalography data. A comprehensive experimental study is conducted in order to comply the final goal, which is to evaluate the actual uses of current means over the area of intereset such as medicine, where decoding an electroencephalography, both efficiently and effectively, could enhance the available brain-computer interfaces and utilize their potential in the real world environments that requiere critical decision making and real-time responses.

\chapter*{}
\thispagestyle{empty}

\noindent\rule[-1ex]{\textwidth}{2pt}\\[4.5ex]

Yo, \textbf{\myName}, alumno de la titulación \myDegree de la \textbf{\myFaculty}, con DNI Y1680851W, autorizo la
ubicación de la siguiente copia de mi Trabajo Fin de Grado en la biblioteca del centro para que pueda ser
consultada por las personas que lo deseen.

\vspace{6cm}

\noindent Fdo: \myName

\vspace{2cm}

\begin{flushright}
Granada a 7 de septiembre de 2018.
\end{flushright}


\chapter*{}
\thispagestyle{empty}

\noindent\rule[-1ex]{\textwidth}{2pt}\\[4.5ex]

D. \textbf{\myProf}, Profesor del Área de Ciencias de la Computación del  \myDepartment de la Universidad de Granada.

\vspace{0.5cm}

\textbf{Informa:}

\vspace{0.5cm}

Que el presente trabajo, titulado \textit{\textbf{\myTitle}},
ha sido realizado bajo su supervisión por \textbf{\myName}, y autoriza la defensa de dicho trabajo ante el tribunal que corresponda.

\vspace{0.5cm}

Y para que conste, expide y firma el presente informe en Granada a 7 de septiembre de 2018.

\vspace{1cm}

\textbf{El director:}

\vspace{5cm}

\noindent \textbf{\myProf}

\chapter*{Agradecimientos}
\thispagestyle{empty}

       \vspace{1cm}


A mi padre, por ser mi padre particular de la informática. A mi madre, por su apoyo incondicional desde siempre. A mi hermano, por alentarme a superarme día a día. A mis más cercanos amigos y compañeros de carrera y profesión, por proveer constante ánimo durante el camino. A mi tutor, por su atención y ayuda a lo largo de todo el desarrollo de este trabajo.

